\section{J\+A\+M\+A Namespace Reference}
\label{namespace_j_a_m_a}\index{J\+A\+M\+A@{J\+A\+M\+A}}


\subsection{Detailed Description}
Cholesky decomposition class

For a symmetric, positive definite matrix A, the Cholesky decomposition is an lower triangular matrix L so that A = L$\ast$\+L'.

If the matrix is not symmetric or positive definite, the constructor returns a partial decomposition and sets an internal flag that may be queried by the is\+S\+P\+D() method.

\begin{DoxyAuthor}{Author}
Paul Meagher 

Michael Bommarito 
\end{DoxyAuthor}
\begin{DoxyVersion}{Version}
1.\+2
\end{DoxyVersion}
Class to obtain eigenvalues and eigenvectors of a real matrix.

If A is symmetric, then A = V$\ast$\+D$\ast$\+V' where the eigenvalue matrix D is diagonal and the eigenvector matrix V is orthogonal (i.\+e. A = V.\+times(D.\+times(V.\+transpose())) and V.\+times(V.\+transpose()) equals the identity matrix).

If A is not symmetric, then the eigenvalue matrix D is block diagonal with the real eigenvalues in 1-\/by-\/1 blocks and any complex eigenvalues, lambda + i$\ast$mu, in 2-\/by-\/2 blocks, [lambda, mu; -\/mu, lambda]. The columns of V represent the eigenvectors in the sense that A$\ast$\+V = V$\ast$\+D, i.\+e. A.\+times(\+V) equals V.\+times(\+D). The matrix V may be badly conditioned, or even singular, so the validity of the equation A = V$\ast$\+D$\ast$inverse(V) depends upon V.\+cond().

\begin{DoxyAuthor}{Author}
Paul Meagher  P\+H\+P v3.\+0 
\end{DoxyAuthor}
\begin{DoxyVersion}{Version}
1.\+1
\end{DoxyVersion}
For an m-\/by-\/n matrix A with m $>$= n, the L\+U decomposition is an m-\/by-\/n unit lower triangular matrix L, an n-\/by-\/n upper triangular matrix U, and a permutation vector piv of length m so that A(piv,\+:) = L$\ast$\+U. If m $<$ n, then L is m-\/by-\/m and U is m-\/by-\/n.

The L\+U decompostion with pivoting always exists, even if the matrix is singular, so the constructor will never fail. The primary use of the L\+U decomposition is in the solution of square systems of simultaneous linear equations. This will fail if is\+Nonsingular() returns false.

\begin{DoxyAuthor}{Author}
Paul Meagher 

Bartosz Matosiuk 

Michael Bommarito 
\end{DoxyAuthor}
\begin{DoxyVersion}{Version}
1.\+1  P\+H\+P v3.\+0
\end{DoxyVersion}
For an m-\/by-\/n matrix A with m $>$= n, the Q\+R decomposition is an m-\/by-\/n orthogonal matrix Q and an n-\/by-\/n upper triangular matrix R so that A = Q$\ast$\+R.

The Q\+R decompostion always exists, even if the matrix does not have full rank, so the constructor will never fail. The primary use of the Q\+R decomposition is in the least squares solution of nonsquare systems of simultaneous linear equations. This will fail if is\+Full\+Rank() returns false.

\begin{DoxyAuthor}{Author}
Paul Meagher  P\+H\+P v3.\+0 
\end{DoxyAuthor}
\begin{DoxyVersion}{Version}
1.\+1
\end{DoxyVersion}
For an m-\/by-\/n matrix A with m $>$= n, the singular value decomposition is an m-\/by-\/n orthogonal matrix U, an n-\/by-\/n diagonal matrix S, and an n-\/by-\/n orthogonal matrix V so that A = U$\ast$\+S$\ast$\+V'.

The singular values, sigma[\$k] = S[\$k][\$k], are ordered so that sigma[0] $>$= sigma[1] $>$= ... $>$= sigma[n-\/1].

The singular value decompostion always exists, so the constructor will never fail. The matrix condition number and the effective numerical rank can be computed from this decomposition.

\begin{DoxyAuthor}{Author}
Paul Meagher  P\+H\+P v3.\+0 
\end{DoxyAuthor}
\begin{DoxyVersion}{Version}
1.\+1
\end{DoxyVersion}
Error handling \begin{DoxyAuthor}{Author}
Michael Bommarito 
\end{DoxyAuthor}
\begin{DoxyVersion}{Version}
01292005
\end{DoxyVersion}
Pythagorean Theorem\+:

a = 3 b = 4 r = sqrt(square(a) + square(b)) r = 5

r = sqrt(a$^\wedge$2 + b$^\wedge$2) without under/overflow. 